\documentclass[a4paper,10pt]{article}
\usepackage[utf8x]{inputenc}
\usepackage{fullpage}
\usepackage[parfill]{parskip}
\usepackage{multicol}
\usepackage{url}
\usepackage[hmargin=.5in,vmargin=.5in]{geometry}


%opening

\begin{document}
\begin{center}

\textbf{SUNY Albany}

\textbf{AMAT592 Fall 2022}

\textbf{Machine Learning}


\end{center}

\textbf{Instructor:} Felix Ye @Catskill 383

\textbf{Lectures:} TTh 10:30am - 11:50am  @ Massry Ctr for Business 231.

\textbf{Instructor Office Hours:}  T 12:30pm-1:20pm, Th 3pm-4pm @ Catskill 330


\textbf{E-Mail Address:} xye2@albany.edu\\
Email will be a major line of communication between the student and the instructor. I will send urgent announcements and important information via email. Please check your university email regularly.

\textbf{Web Page:}
Check the course page in blackboard regularly. Homework assignments, course announcements, and grades will be posted there.


\textbf{Course Description:} The primary goal is to provide students with the tools and principles needed to solve both the traditional and modern data science problems. In particular, the course covers a wide variety of topics in machine learning. It introduces the key terms, concepts and methods in machine learning, with an emphasis on developing critical analytical skills through hands-on exercises of data analysis tasks. In addition, the students will practice basic programming skills to use software tools in machine learning. The programming language in this course will be Jupyter notebook. 

All latest jupyter notebook can be downloaded in https://github.com/yexf308/MAT592.git. I will keep updating this git repository as class progresses, so please fetch the update regularly. I will also leave a copy (not most updated) version in blackboard. 

 Prerequisite: vector calculus, probability, linear Algebra, optimization and some experience on programming. 
 

\textbf{Textbook:} Probabilistic Machine Learning: An Introduction \\
You can download the draft of the latest version: https://probml.github.io/pml-book/book1.html. 
 I will loosely follow this book, however, I will put some additional material into it.


\textbf{Grading Policy:}

\begin{tabular}{lr}
Homework & 100\%\\
\end{tabular}




Incomplete grade: This class will not give any
incomplete grade. If the work cannot complete in the
current semester, the student can choose to retake this
class in the following year.






\textbf{Homework:} Homework assignments will be assigned every three weeks. There are 5 sets of homework in total. 

Late assignment turn-in is not permitted. Any assignment turned in after the deadline will NOT be graded.




\textbf{Attendance:} Although attendance will not be taken, I strongly encourage you attend and participate in every lecture. This is one of the best ways to ensure success in the course.






\textbf{Academic Misconduct:} The strength of the university depends on academic and personal integrity. In this course, you must be honest 
and truthful. Ethical violations include cheating on exams, plagiarism, reuse of assignments, improper use 
of the Internet and electronic devices, unauthorized collaboration, alteration of graded assignments, forgery 
and falsification, lying, facilitating academic dishonesty, and unfair competition.

In addition, specific ethics guidelines for this course are as follows: 
\begin{itemize}
\item Homeworks must be done individually: each student must hand in their own answers. In addition, each student must write their own code in the programming part of the assignment. It is acceptable, however, for students to collaborate in figuring out answers and helping each other solve the problems.

\item To be more precise, on every homework:
list every person with whom you discussed any problem in any depth, and every reference (outside of our course slides, lectures, and textbook) that you used.

\item You can spend an arbitrary amount of time discussing and working out a solution with your listed collaborators, but *do not take notes, photos, or other artifacts of your collaboration*. Erase the board you were working on, and once you're alone, write up your answers yourself.
This means that word-for-word phrases shared between homeworks will leave us with a high degree of suspicion that the whiteboard policy was not followed, and similarly high degrees of similarity between programming assignments. We do scan for the latter programmatically, and the former manually.

\item The homework problems have been carefully chosen for their pedagogical value and hence might be similar or identical to those given out in past offerings of this course at SUNY Albany, or similar courses at other schools. Using any pre-existing solutions from these sources, from the Web or other textbooks constitues a violation of the academic integrity expected of you and is strictly prohibited.
\end{itemize}

If you are caught copying and pasting someone else’s code the following penalty system will apply:
\begin{itemize}
\item For the first offense, a zero for the question that you copied on.
\item For the second offense, a zero for the assignment and I file a Violation of Academic Integrity Report (VAIR).
\item For the third or later offense, you get a letter grade reduction (or possibly fail) and I refer you to community standards, which could result in expulsion from UAlbany.
\end{itemize}





\end{document}
