\documentclass[a4paper,10pt]{article}
\usepackage[utf8x]{inputenc}
\usepackage{fullpage}
\usepackage[parfill]{parskip}
\usepackage{multicol}
\usepackage{url}
\usepackage[hmargin=.5in,vmargin=.5in]{geometry}


%opening

\begin{document}
\begin{center}

\textbf{SUNY Albany}

\textbf{MAT592 Fall 2021}

\textbf{Machine Learning}


\end{center}

\textbf{Instructor:} Felix Ye 

\textbf{Lectures:} TuTh 10:30am - 11:50am (session 1), 1:30PM - 2:50PM (session 2) in Massry Ctr for Business 368.

\textbf{Instructor Office Hours:} Th 3:00pm-4:00pm in ES 123 or by appointment. 


\textbf{E-Mail Address:} xye2@albany.edu\\
Email will be a major line of communication between the student and the instructor. I will send urgent announcements and important information via email. Please check your university email regularly.

\textbf{Web Page:}
Check the course page in blackboard regularly. Homework assignments, course announcements, and grades will be posted there.


\textbf{Course Description:} The primary goal is to provide students with the tools and principles needed to solve both the traditional and modern data science problems. In particular, the course covers a wide variety of topics in machine learning. It introduces the key terms, concepts and methods in machine learning, with an emphasis on developing critical analytical skills through hands-on exercises of data analysis tasks. In addition, the students will practice basic programming skills to use software tools in machine learning. The programming language in this course will be Jupyter notebook. 

All latest jupyter notebook can be downloaded in https://github.com/yexf308/MAT592.git. I will keep updating this git repository as class progresses, so please fetch the update regularly. I will also leave a copy (not most updated) version in blackboard. 

 Prerequisite: vector calculus, probability, linear Algebra, optimization and some experience on programming. 
 

\textbf{Textbook:} Probabilistic Machine Learning: An Introduction \\
You can download the draft of the latest version: https://probml.github.io/pml-book/book1.html. 
 I will loosely follow this book, however, I will put some additional material into it.


\textbf{Grading Policy:}

\begin{tabular}{lr}
Homework & 70\%\\ 
Take-home Final & 30\% \\
\end{tabular}

Incomplete grade: This class will not give any
incomplete grade. If the work cannot complete in the
current semester, the student can choose to retake this
class in the following year.




\textbf{Exams:} There will be one take-home final. The detail will be given at the end of the semester. 


\textbf{Homework:} Homework assignments will be assigned each month. There are 3 sets of homework in total. 

Late assignment turn-in is not permitted. Any assignment turned in after the deadline will NOT be graded.




\textbf{Attendance:} Although attendance will not be taken, I strongly encourage you attend and participate in every lecture. This is one of the best ways to ensure success in the course.






\textbf{Academic Misconduct:} The strength of the university depends on academic and personal integrity. In this course, you must be honest 
and truthful. Ethical violations include cheating on exams, plagiarism, reuse of assignments, improper use 
of the Internet and electronic devices, unauthorized collaboration, alteration of graded assignments, forgery 
and falsification, lying, facilitating academic dishonesty, and unfair competition.

In addition, specific ethics guidelines for this course are as follows: Students may discuss homework. However, 
all solutions MUST be written up and submitted individually. The same rules apply to computer programs. 
Basic ideas may be discussed but detailed codes should not be copied or shared. Finally, exams must 
represent the result of individual effort and communication is permitted only with the instructor.

Report any violations you witness to the instructor. You may consult the associate dean of student affairs 
and/or the chairman of the Ethics Board beforehand. 

\textbf{Tentative Course Outline and Schedule:}


\begin{itemize}
\item Week 1 (Aug 24 \& 26): Review of Pre-requisite: linear algebra and optimization. 

\item Week 2 (Aug 31 \& Sep 2): Review of Pre-requisite: probability. Machine Learning: Overview.

\item Week 3 (Sep 7 \& 9): Linear regression, Logistic regression. Homework 1 assigned. 

\item Week 4 (Sep 14 \& 16): More on regression.

\item Week 5 (Sep 21 \& 23): Support Vector Machines, Kernel method. 

\item Week 6 (Sep 28 \& 30): Binary classification revisited. Homework 1 due. Homework 2 assigned.  

\item Week 7 (Oct 5 \& 7): KNN classification.  

\item Week 8 (Oct 14): Decision tree.  

\item Week 9 (Oct 19 \& 21): Boosting methods. Homework 2 due. Homework 3 assigned.   

\item Week 10 (Oct 26 \& 28): K-means clustering.

\item Week 11 (Nov 2 \& 4): Gaussian mixture model.

\item Week 12 (Nov 9 \& 11): Principal component analysis.  Homework 3 due. Take-home final is assigned. 

\item Week 13 (Nov 16 \& 18):  Deep learning. 

\item Week 14 (Nov 23): Deep learning. 

\item Week 15 (Nov 30 \& Dec 2): Convolutional neural networks \& Recurrent neural networks. Take-home final due.
 
\end{itemize}






\end{document}
